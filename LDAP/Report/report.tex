\documentclass[a4paper,14pt]{article}
\usepackage{amsmath}
\usepackage[utf8]{inputenc} % 
\usepackage{graphicx}
\graphicspath{{pic/}}
\DeclareGraphicsExtensions{.png,.jpg}
\usepackage{multicol}

\usepackage[russian]{babel} % правила переноса
\usepackage[left=2cm,right=2cm,
top=2cm,bottom=2cm,bindingoffset=0cm]{geometry} % для изменения размеров полей документа
\usepackage{commath}
\usepackage{listings}
% \usepackage[framed,numbered,autolinebreaks,useliterate]{mcode}
% \usepackage{longtable}

\begin{document}

%%%%%%%%%%%%%%%%%%%%%% Титульный лист %%%%%%%%%%%%%%%%%%%%%%

\begin{titlepage}
	\newpage
	\begin{center} % Размещение ткста - по центру
		Санкт-Петербургский политехнический\\ 
		университет Петра Великого\\
		Институт компьютерных наук и технологий\\
		Кафедра компьютерных систем и программных технологий\\
		\vspace{7cm}
		\textbf {Отчёт по лабораторной работе}\\
		\textbf {Дисциплина:} Сети и телекоммуникационные технологии\\
		\textbf{Тема:} Организация сетевого взаимодействия. Протокол TCP
	\end{center} % Конец размещения
	\vspace{8cm} % 
	
	\vfill
	
	\flushleft{Выполнил студент группы 43501/1} 
	\hfill\parbox{9 cm}{\hspace*{3cm}\hbox to 0cm{\raisebox{-1em}{\small(подпись)}}\hspace*{-0.8cm}\rule{3cm}{0.8pt} К.О. Поляков}\\[0.6cm]
	
	\flushleft{Преподаватель} \hfill\parbox{9 cm}{\hspace*{3cm}\hbox to 0cm{\raisebox{-1em}{\small(подпись)}}\hspace*{-0.8cm}\rule{3cm}{0.8pt} А.О.Алексюк }\\[0.6cm]
	
	\hfill\parbox{9 cm}{\hspace*{5cm} \today }\\[0.6cm]

	\vspace{\fill}
	\begin{center}
		Санкт-Петербург \\ 2017	
	\end{center}
\end{titlepage}

\tableofcontents

\newpage

\setcounter {section}{0}
\setcounter {equation}{0}
\setcounter {figure}{0}
\section{Цель работы}
\hspace{0,5cm}   Изучение принципов программирования сокетов с использованием протокола TCP.
\section{Индивидуальное задание}
разработать приложение–клиент и приложение–сервер службы каталогов. Всякая запись в каталоге состоит из одного или не- скольких атрибутов и обладает уникальным именем. Уникальное имя со- стоит из одного или нескольких относительных уникальных имен, разде- лённых запятой (например, “ cn=Users, dc=myserver, dc=myprovider, dc=ru ”).

Основные возможности:
Серверное приложение должно реализовывать следующие функции:
\begin{enumerate}
\item Прослушивание определенного порта
\item Обработка запросов на подключение по этому порту клиентов
\item Поддержка одновременной работы нескольких клиентов с использованием механизма нитей и средств синхронизации доступа к разделяемым между нитями ресурсам.
\item Прием запросов на поиск, добавление и удаление записей службы
каталогов
\item Осуществление поиска, добавления, удаления записей службы ката-
логов
\item Передача клиенту записей службы каталогов и подтверждений о вставке и удалении записей
\item Обработка запроса на отключение клиента
\end{enumerate}
Клиентское приложение должно реализовывать следующие функции:
	\begin{enumerate}
\item Возможность параллельной работы нескольких клиентов с одного или нескольких IP-адресов
\item Установление соединения с сервером (возможно, с регистрацией на сервере)
\item Разрыв соединения
\item Передача запросов о поиске, добавлении, удалении записей серверу
\item Передача команды на удаление валюты
\item Разрыв соединения
\item Обработка ситуации отключения клиента сервером
\end{enumerate}
Настройка приложений:

Разработанное клиентское приложение должно предоставлять пользователю возможность задания IP-адреса или доменного имени сервера, а также номера порта сервера.


\section{Разработанный прикладной протокол}
Протокол TCP имеет следующий шаблон сообщения:
\begin{center}
	\textbf{<команда> <аттрибут> }
\end{center}
В начале сообщения, всегда присутствует тип команды, далее взависимости от команды могут идти(взависимости от типа команды) аттрибуты, которые отделены друг от друга пробелом.

Список команд, которыми оперирует клиент:
\begin{lstlisting}
| Add New Entity:               | add <path>
| Delete Entity:                | del <path>  
| Read All Entities:            | all                      
| Find full path for Entity:    | fnd <part of path>  
| Show help:                    | help 
| Exit :                        | exit 
\end{lstlisting}

\subsection{Описание структуры приложения}
Сервер:\\
Функция main:
	\begin{itemize}		
        \item Инициализация всех используемых переменных;
        \item Запуск событий, если таковые уже имеются;
        \item Создание сокета;
        \item Создание потока для прослушивания сокета;
	\end{itemize}
 Поток для прослушивания сокета:
    \begin{itemize}
        \item Прослушивание сокета;
        \item Добавление подключенного клиента в список;
        \item Создание сокета для работы с клиентом;
        \item Создание потока для работы с клиентом.
	\end{itemize}
Цикл чтения команд:
    \begin{itemize}
        \item Чтение команды из стандартного ввода;
        \item Реакция на введенную команду (при корректном вводе команды) или вывод предупреждения.
    \end{itemize}


Клиент:\\
Функция main:
	\begin{itemize}	
\item Чтение ip адреса сервера;
\item Подключение к серверу;
\item Создание потока для получения данных от сервера;
\item Цикл чтения данных и отправка серверу.
\end{itemize}
Поток для получения данных от сервера:
 \begin{itemize}
\item Побайтовое получение данных от сервера.
	\end{itemize}
\section{Реализация программы}
\subsection{Структура проекта}
При разработке приложения для операционной системы семейства Windows использовалась среда разработки CLion.

Язык программирования — С.

\subsection{Сетевая часть TCP}
Клиентское приложение в TCP только отсылает команды на сервер, поэтому оно ничем не отличается от telnet клиента. Сервер обрабатывает команды, работает с коллекциями, сохраняет и загружает свое состояние, присылает уведомления и др. Делаем вывод, что клиентская программа потребляет ничтожно малый процессорный ресурс, в то время как сервер - наоборот.

На сервере, в первую очередь, происходит инициализация WinSock (на Windows), создание сокета (функция socket), привязка сокета к конкретному адресу (функция bind). 

После этого ожидаем подключения клиентов в бесконечном цикле, с помощью функции accept. Если функция возвращает положительное значение, которое является клиентским сокетом, то создаем новый поток, в котором обрабатываем клиентские сообщения. 

Клиентский поток вызывает функцию считывания символов в бесконечном цикле, как только при считывается знак перевода строки, функция возращает прочитанные символы. Если функция не вернула исключение, то посылаем команду на обработку, в противном случае это обозначает отключение клиента. 

\section{Вывод}
В ходе работы были изучены принципы программирования сокетов с использованием протокола TCP.\\
В рамках модели клиент (linux) и сервер (windows). Были изучены и использованы разные библеотеки для построения работы сокетов. Серверное прилоежние имеет несколько потоков-обработчиков, для успешной работы с несколькими клиентами единовременно. Протокол TCP упростил логику приложения, переложив работу с неправильной передачей данных на себя. 
Так же в ходе разработки приложения была решениа проблема синхронизации данных в многопоточной среде.То есть были получены навыки организации многопоточного сервера, изучены принципы синхронизации доступа к глобальным переменным.\\
В разработанном в ходе работы сервере для каждого клиента создается отдельный поток. Такой подход оправдан, т.к. клиенты могут исполнять долгие операции и операции различной трудоемкости. В этом случае использование отдельного потока для каждого клиента обеспечивает минимизацию взаимного влияния клиентов друг на друга.  \\\\

\end{document}